\documentclass{article}
\title{CSE 847 Homework One}
\author{Reuben Lewis}
\date{05 February, 2021}

\usepackage{amsfonts}
\usepackage{amsmath}

\newcommand{\R}{{\rm\hbox{I\kern-.15em R}}}
\newcommand{\IR}{{\rm\hbox{I\kern-.15em R}}}
\newcommand{\IN}{{\rm\hbox{I\kern-.15em N}}}
\newcommand{\IZ}{{\sf\hbox{Z\kern-.40em Z}}}
\newcommand{\IS}{{\rm\hbox{S\kern-.45em S}}}
\newcommand{\Real}{I\!\!R}


\begin{document}
    \maketitle
    \section{Introduction}
    \subsection{Pg 58, 1.3}
        \begin{tabular}{ l | l | l | l | l}
            Box & Apples & Oranges & Limes & Selection Prob $p$ \\
            \hline
            tab$r$ & 3 & 4 & 3 & 0.2\\
            $b$ & 1 & 1 & 0 & 0.2\\
            $g$ & 3 & 3 & 4 & 0.6\\
        \end{tabular}

        (1) What is the probability of selecting is an apple?
            $p(A) = p(A,r) + p(A,b) + p(A,g)$ \\
            $p(A) = p(A|r)*p(r) + p(A|b)*p(b) + p(A|g)*p(g)$ \\
            $p(A) = 0.3*0.2 + 0.5*0.2 + 0.3*0.6$ \\
            $p(A) = 0.34$ \\

        (2) If the selected fruit is an orange, what is the probability 
            it came from the green box? \\
            $p(O) = p(O,r) + p(O,b) + p(O,g)$ \\
            $p(O) = p(O|r)*p(r) + p(O|b)*p(b) + p(O|g)*p(g)$ \\
            $p(O) = 0.4*0.2 + 0.5*0.2 + 0.3*0.6$ \\
            $p(O) = 0.36$ \\
            $p(g|O) = \frac{p(O|g)p(g)}{p(O)}$ \\
            $p(g|O) = \frac{0.3 * 0.6}{0.34}$ \\
            $p(g|O) = 0.5$ 

    \subsection{Pg 59, 1.6}
    $cov[x,y] = 0$\\
    $\mathbb{E} [xy] - \mathbb{E}[x] \mathbb{E}[y] = 0$\\
    $\mathbb{E} [xy] = \mathbb{E}[x] \mathbb{E}[y]$\\
    $\sum_{x,y} xy P_{x,y}(x,y)= \sum_x x P(x) * \sum_y y P(y)$\\
    $P(x,y)= P(x) * P(y)$\\
    This is the definition of independence, so, if $x$ and $y$ are independent,
    their covariance is going to be $0$.
    
    \subsection{Pg 59, 1.11}
    Solve first for $\mu_{ML}$, then for $\sigma^{2}$ to verify the results of (1.55) and (1.56). This
    involves first taking the partial derivative with respect to the value we are solving for, then
    setting that equation to $0$ and solving for the value.\\
    (1) $\mu_{ML}$ \\
    \begin{center}
        $-\frac{1}{2\sigma^2} \sum_{n-1}^{N}(x_n -\mu)^2 - \frac{N}{2}\ln(\sigma^2) - \frac{N}{2}\ln(2\pi)$\\
        $\frac{\partial}{\partial \mu} \Rightarrow -\frac{1}{2\sigma^2} \sum_{n-1}^{N}\frac{\partial}{\partial \mu} (x_n -\mu)^2$\\
        $-\frac{1}{2\sigma^2} \sum_{n-1}^{N} 2 * -1 * (x_n -\mu)$\\
        $\frac{1}{\sigma^2} \sum_{n-1}^{N} (x_n -\mu)$\\
        $0 = \frac{1}{\sigma^2} \sum_{n-1}^{N} (x_n -\mu)$\\
        $0 = \sum_{n-1}^{N} (x_n -\mu)$\\
        $0 = \sum_{n-1}^{N} x_n - N*\mu$\\
        $N * \mu = \sum_{n-1}^{N} x_n$\\
        $\mu_{ML} = \frac{1}{N}\sum_{n-1}^{N} x_n$\\
    \end{center}
    (2) $\sigma^2$ \\
    \begin{center}
        $-\frac{1}{2\sigma^2} \sum_{n-1}^{N}(x_n -\mu)^2 - \frac{N}{2}\ln(\sigma^2) - \frac{N}{2}\ln(2\pi)$\\
        $\frac{\partial}{\partial \sigma^2} \Rightarrow -\frac{\partial}{\partial \sigma^2}\frac{1}{2\sigma^2} \sum_{n-1}^{N} (x_n -\mu)^2 - \frac{\partial}{\partial \sigma^2}\frac{N}{2}\ln(\sigma^2)$\\
        $\frac{1}{\sigma^4} \sum_{n-1}^{N} (x_n -\mu)^2 - \frac{N}{2\sigma^2}$\\
        $0 = \frac{1}{2\sigma^4} \sum_{n-1}^{N} (x_n -\mu)^2 - \frac{N}{2\sigma^2}$\\
        $\frac{N}{2\sigma^2} = \frac{1}{2\sigma^4} \sum_{n-1}^{N} (x_n -\mu)^2 $\\
        $N*\sigma^2 = \sum_{n-1}^{N} (x_n -\mu)^2 $\\
        $\sigma^2 = \frac{1}{N}\sum_{n-1}^{N} (x_n -\mu)^2 $\\
    \end{center}

    \section{Linear Algebra}
    \begin{enumerate}
        \item Matrix Calculations
        $$ 
        A = \begin{bmatrix} 1 & 2& 3\\ 2 & 1 & 4 \end{bmatrix},
        B = \begin{bmatrix} 1 & 0 \\ 2 & 1 \\ 3 & 2\end{bmatrix}, 
        C = \begin{bmatrix} 3 & -1 & 3 \\ 4 & 1 & 5\\ 2 & 1 & 3 \end{bmatrix}, 
        D = \begin{bmatrix} 2 & -4 & 5 \\ 0 & 1 & 4\\ 3 & 2 & 1 \end{bmatrix}, 
        E = \begin{bmatrix} 3 & -2 \\ 2 & 4 \end{bmatrix}.
        $$
        \begin{enumerate}
            \item $(2A)^T$
                First, scale $A$ by 2, then transpose it.
                \begin{center}
                    $2*A = \begin{bmatrix} 2 & 4& 6\\ 4 & 2 & 8 \end{bmatrix}$\\
                    $(2*A)^T = \begin{bmatrix} 2 & 4 \\ 4 & 2\\ 6 & 8 \end{bmatrix}$
                \end{center}
            \item $(A-B)^T$
                Because $A$ and $B$ don't have the same dimension, this equation doesn't work. $A - B^T$ would, however, since $B$ would
                then have the same dimensions as $A$.
            \item $(3B^T - 2A)^T$
                First, scale the matrices, then apply the transpose onto $B$. Next, subtract the matrices and transpose that output.
                \begin{center}
                    $2*A = \begin{bmatrix} 2 & 4& 6\\ 4 & 2 & 8 \end{bmatrix}$\\
                    $3*B = \begin{bmatrix} 3 & 0\\ 6 & 3 \\ 9 & 6 \end{bmatrix}$\\
                    $3*B^T = \begin{bmatrix} 3 & 6 & 9\\ 0 & 3 & 6 \end{bmatrix}$\\
                    $3*B^T - 2*A = \begin{bmatrix} 1 & 2 & 3\\ -4 & 1 & -2 \end{bmatrix}$\\
                    $(3*B^T - 2*A)^T = \begin{bmatrix} 1 & -4 \\ 2 & 1\\ 3 & -2 \end{bmatrix}$\\
                \end{center}
            \item $(-A)^T E$ 
                First, distribute the $-1*A$, and then transpose it. Next, multiply it by $E$, since there inner dimensions match (3x2 * 2x2).\\
                \begin{center}
                    $-1*A = \begin{bmatrix} -1 & -2& -3\\ -2 & -1 & -4 \end{bmatrix}$
                    $(-1*A)^T = \begin{bmatrix} -1 & -2 \\ -2 & -1\\  -3 & -4 \end{bmatrix}$
                    $(-1*A)^T E = \begin{bmatrix} -1*3 + -2*2 & -2*-1  + -2*4\\ -2*3 + -1*2  & -2*2 + -1*4  \\ -3*3 + -4*2 & -3*-2 + -4*4\end{bmatrix}$
                    $(-1*A)^T E = \begin{bmatrix} -7 & -6 \\ -8  & 0  \\ -17 & -10 \end{bmatrix}$
                \end{center}
            \item $(C+ D^T + E)^T$\\
                Because $E$ is not the same dimensions as $C$ or $D$, it can not be added to the others, so this operation cannot be done.
        \end{enumerate}
        \item Subspace of $\IR^2$? Justify your answer. 
            \begin{enumerate}
                \item $\{(x,y ) \in \IR^2 | x^2 + y^2 = 0 \}$\\
                    This is not a subspace, since two vectors $(1,0) + (1,1) = (2,1)$ and show's it's not closed under addition. 
                \item $\{(x,y ) \in \IR^2 | x^2 - y^2 = 0 \}$\\
                    This is not a subspace, since two vectors $(1,1) + (1,-1) = (2,0)$ and show's it's not closed under addition. 
                \item $\{(x,y ) \in \IR^2 | x^2 - y = 0 \}$\\
                    This is not a subspace, since two vectors $(1,1) + (1,-1) = (2,0)$ and show's it's not closed under addition. 
                \item $\{(x,y ) \in \IR^2 | x - y = 0 \}$\\
                    This is a subspace, since it satisfies the three main properties. The $0$ vector is a part of the space,
                    it's closed under vector addition, and it's closed under scalar multiplication. This is a special case of a
                    line with a slope of 1.
                    \begin{center}
                        Closed under Vector addition:\\
                        $u = (x_1, y_1), v = (x_2, y_2)$\\
                        $u + v = (x_1+x_2, y_1+y_2)$\\
                        $u + v = (x_1+x_2, x_1+x_2)$\\
                    \end{center}
                    This process is similar for scaling as well.
                \item $\{(x,y ) \in \IR^2 | x - y = 1 \}$\\
                    This is not a subspace because it doesn't contain the origin.
            \end{enumerate}
        \item $$ A = \begin{bmatrix}1 & 2\\ 3 & 2\end{bmatrix}, B =  \begin{bmatrix}2 & -1\\ -3 & 4\end{bmatrix}.$$
        Is $AB = BA$?
        No.\\
        \begin{center}
            $AB =  \begin{bmatrix}1*2 + 2*-3 & 1*-1 + 2*4\\ 3*2 + 2*-2 & 3*-1 + 2*4\end{bmatrix}$\\
            $AB =  \begin{bmatrix}-4 & 7\\ 2 & 5\end{bmatrix}$\\
            $BA =  \begin{bmatrix}2*1 + -1*3 & 2*2 + -1*2\\ -3*1 + 4*3 & -3*2 + 4*2\end{bmatrix}$\\
            $BA =  \begin{bmatrix}-1 & 2\\ 9 & 2\end{bmatrix}$
        \end{center}
        \item
        \begin{enumerate}
            \item Because of how matrix multiplication works, you multiply the rows of the first matrix by the columns of the
            second. This means that if there is a row of 0s in the first matrix, that row would be multiplied against every 
            column of the second matrix. This would cause a row of 0s in the resulting matrix, since multiplying any value
            by 0 results in 0. As an example, imagine $A = 2 X 3$ matrix and $B = 3 X 4$, with a row of 0s at $i = 2$. Row 2 
            of the resulting matrix would be calculated like so:
            \begin{center}
                $A_{2k} = \sum_{j} 0_{2,j} * b_{j,k}$\\
                $\begin{bmatrix}0*b_{0,0} + 0*b_{1,0} + 0*b_{2,0} & 0*b_{0,1} + 0*b_{1,1} + 0*b_{2,1}\end{bmatrix}$\\
                $\begin{bmatrix}0 & 0 \end{bmatrix}$\\
            \end{center}
            \item The best way to show this is through an example. Let
            \begin{center}
                $$ 
                A = \begin{bmatrix} 1 & 0\\ 1 & 0 \end{bmatrix},
                B = \begin{bmatrix} 1 & 2 \\ 3 & 4 \end{bmatrix}
                $$
            \end{center}
            If we multiply $BA$, then we get the following matrix. This is because when we multiply rows of $B$ through the
            columns of $A$, the column of 0s in $A$ nullifies the values in the row of $B$, resulting in a column of 0s in 
            the resulting matrix.
            \begin{center}
                $BA = \begin{bmatrix} 1*1 + 2*1 & 1*0  + 2*0 \\ 3*1 + 4*1 & 3*0  + 4*0 \end{bmatrix}$\\
                $BA = \begin{bmatrix} 3 & 0 \\ 7 & 0 \end{bmatrix}$\\
            \end{center}
        \end{enumerate}
        \item Show that $\mbox{rank}(AB) \leq \min\{\mbox{rank}(A), \mbox{rank}(B)\}$.
        \begin{center}
            Remember that $rank(AB) = dim(ran(AB)), rank(A) = dim(ran(A))$
            By definition, $ran(AB) = {y \in \IR^n : y = ABx for some x \in \IR^n}$
            Substitute $z = Bx \Rightarrow y = A(Bx) \Rightarrow {y \in \IR^n : y = Az for some x \in \IR^n} = ran(a)$
        \end{center}
        This shows that the vector space $ran(AB)$ is a subspace of $ran(A)$, which allows us to say 
        $rank(AB) = dim(ran(AB)) \leq dim(ran(A)) = rank(A)$. Now what about $rank(B)$? 
        \begin{center}
            Remember that $rank(A) = rank(A^T)$
            Let $C = A^T$ and $D = B^T$
            $rank(DC) \leq rank(D) \Rightarrow rank(C^TD^T) \leq rank(D^T)$
            Translating that back over, we get $rank(AB) \leq rank(B)$
        \end{center}
        So, by that logic, $rank(AB)$ is less than or equal to whichever of $A$ or $B$ has the lower
        rank.
    \end{enumerate}

\end{document}